%%%%%%%%%%%%%%%%%%%%%%%%%%%%%%%%%%%%%%%%%%%%%%%%%%%%%%%%%%%%%%%%%%%%%%%%%%%%%%%%
%2345678901234567890123456789012345678901234567890123456789012345678901234567890
%        1         2         3         4         5         6         7         8

\documentclass[letterpaper, 12 pt, conference]{ieeeconf}  % Comment this line out
                                                          % if you need a4paper
%\documentclass[a4paper, 12pt, conference]{ieeeconf}      % Use this line for a4
                                                          % paper

\IEEEoverridecommandlockouts                              % This command is only
                                                          % needed if you want to
                                                          % use the \thanks command
\overrideIEEEmargins
% See the \addtolength command later in the file to balance the column lengths
% on the last page of the document

\usepackage{hyperref}
\usepackage[utf8]{inputenc}
\usepackage{enumerate}
\usepackage{natbib}
\usepackage{graphicx}
\usepackage[spanish]{babel}
\hypersetup{
    colorlinks=true,
    linkcolor=blue,
    filecolor=magenta,      
    urlcolor=cyan,
}

% The following packages can be found on http:\\www.ctan.org
%\usepackage{graphics} % for pdf, bitmapped graphics files
%\usepackage{epsfig} % for postscript graphics files
%\usepackage{mathptmx} % assumes new font selection scheme installed
%\usepackage{times} % assumes new font selection scheme installed
%\usepackage{amsmath} % assumes amsmath package installed
%\usepackage{amssymb}  % assumes amsmath package installed

\title{\LARGE \bf
Práctica 6: dispositivos varios 
}

%\author{ \parbox{3 in}{\centering Narshion Ngao*
%         \thanks{*Use the $\backslash$thanks command to put information here}\\
%         Msc. Computer Systems - 2018\\
%         Jomo Kenyatta University of Agriculture \& Technology \\
%       
%}}

\author{Universidad de San Carlos de Guatemala \\% <-this % stops a space
Escuela de Ciencias Físicas y Matemáticas\\
Laboratorio de Circuitos\\
Segundo Semestre 2019
}


\begin{document}



\maketitle
\thispagestyle{empty}
\pagestyle{empty}

\section{Objetivos}
\begin{itemize}
    \item General: experimentar con el uso de dispositivos básicos de utilidad en electrónica y electricidad.
    \item Específicos:
    \begin{enumerate}
    \item Ejercitar habilidades de diseño de circuitos para solución de problemas específicos.
    \item Explorar el uso de optoacopladores para control de circuitos con mayor consumo de corriente.
    \item Comprobar el uso de relés como segundo método de control de acoples eléctricos.
    \item Practicar el diseño fundamental de galvanómetros como voltímetros analógicos básicos.
\end{enumerate}
\end{itemize}


\section{Materiales}
\begin{itemize}
    \item 1 relé SPST 5V.
    \item 1 botón NC para protoboard.
    \item 1 botón NA para protoboard.
    \item 1 optoacoplador 4n25.
    \item 1 batería AA con adaptador o fuente 3.3V.
    \item 1 ventilador DC 12V.
    \item 1 LED RGB.
    \item 1 resistencia ohm 1/4W.
    \item 1 fuente 5V y 12V.
    \item 1 protoboard.
    \item Alambre para protoboard.
    \item 1 galvanómetro 10V o 30V escala máxima.
    \item 1 potenciómetro de precisión 200 ohms o resistencia de precisión.
\end{itemize}
\pagebreak

\section{Procedimiento y reporte de resultados}
Seguir todos los pasos que a continuación se enlistan. Al terminar de armar cada circuito, pedir la evaluación con lista de cotejo.

\begin{enumerate}
    \item Conectar el ventilador a una fuente de voltaje 3.3V y luego con una de 12V. Observar lo que sucede.
    \item Con un optoacoplador diseñar un circuito capaz de encender el ventilador eficientemente cuando se tenga en la entrada un pulso de botón de 3.3V.
    \item Diseñar un circuito que cumpla las siguientes especificaciones.
        \begin{itemize}
            \item La salida normal del circuito debe ser una luz roja.
            \item Al presionar un botón, esta debe cambiar a verde.
            \item Si se presiona un segundo botón, el color rojo debe cambiar a magenta y el verde a cyan.
        \end{itemize}
    \item En los grupos asignados para proyecto, hacer las modificaciones necesarias para cambiar la escala del galvanómetro a 5V máximo. Mostrar el producto final en la siguiente clase.
\end{enumerate}


\addtolength{\textheight}{-12cm}   % This command serves to balance the column lengths
                                  % on the last page of the document manually. It shortens
                                  % the textheight of the last page by a suitable amount.
                                  % This command does not take effect until the next page
                                  % so it should come on the page before the last. Make
                                  % sure that you do not shorten the textheight too much.

\end{document}